%%Common packages and definitions for writing ML/NLP papers
\documentclass[11pt,a4paper]{article}
\usepackage{etex}  % fixes ``No room for a new \dimen...'' error from too many packages.
\usepackage{xparse}  % makes writing commands easier
\usepackage{standalone}
\usepackage{times}
\usepackage{url}
\usepackage{amsmath}
\usepackage{amssymb}
\usepackage{amsfonts}
\usepackage{amssymb}
\usepackage{mathtools}
\usepackage{stmaryrd} % more math symbols
\usepackage{graphicx}
\usepackage{color}
\usepackage{adjustbox}
\usepackage[toc,page]{appendix}
\usepackage{xspace}
\newcommand\bmmax{0} % magic to avoid 'too many math alphabets' error
\usepackage{bm}
\usepackage[shortcuts]{extdash}  % for non-breaking hyphens. Stack Overflow says it's important for this to be the last package loaded.
\usepackage[skip=5pt]{caption}
\usepackage{subcaption}
\usepackage{array}
\usepackage{tabu}
\usepackage{makecell}
\usepackage{paralist}
\usepackage{cases}
\usepackage{diagbox}
\usepackage{enumitem}
\usepackage{soul}
\usepackage{multirow}
\usepackage{verbatim}
\usepackage{tabularx}
\usepackage{tabulary}
\usepackage{booktabs}
\usepackage[mathscr]{euscript}
\usepackage{algorithm}
\usepackage{algpseudocode}
\usepackage{amsthm}
\usepackage{tikz}
\usepackage{tikz-dependency}
\usetikzlibrary{automata,decorations.markings,arrows,positioning,matrix,calc,patterns,angles,quotes,calc}
\usepackage{pgfplots}
\pgfplotsset{compat=1.10}


\algrenewcommand{\algorithmiccomment}[1]{\leavevmode\hfill$\triangleright$ #1}

\definecolor{orange}{rgb}{1,0.5,0}
\definecolor{mdgreen}{rgb}{0.05,0.6,0.05}
\definecolor{mdblue}{rgb}{0,0,0.7}
\definecolor{dkblue}{rgb}{0,0,0.5}
\definecolor{dkgray}{rgb}{0.3,0.3,0.3}
\definecolor{slate}{rgb}{0.25,0.25,0.4}
\definecolor{gray}{rgb}{0.5,0.5,0.5}
\definecolor{ltgray}{rgb}{0.7,0.7,0.7}
\definecolor{purple}{rgb}{0.7,0,1.0}
\definecolor{lavender}{rgb}{0.65,0.55,1.0}


\DeclareMathOperator*{\argmax}{argmax}
\DeclareMathOperator*{\argmin}{argmin}
\DeclareMathOperator*{\softmax}{softmax}
\DeclareMathOperator*{\p}{Pr}
\DeclareMathOperator*{\expectation}{\mathbb{E}}
% Expectation: can use \E{X}  or \E[Y]{X}.(requires xparse)
\NewDocumentCommand{\E}{o m}{%
  \ensuremath{\expectation%
    \IfValueT{#1}{%
      _{#1}%
    }%
    \left[ #2 \right]%
  }%
}
% KL Divergence
\DeclarePairedDelimiterX{\infdivx}[2]{(}{)}{%
  #1\;\delimsize|\delimsize|\;#2%
}
\newcommand{\kld}[2]{\ensuremath{D_{KL}\infdivx{#1}{#2}}}
\DeclarePairedDelimiter{\norm}{\lVert}{\rVert}
\DeclarePairedDelimiter{\abs}{\lvert}{\rvert}%
\DeclarePairedDelimiter{\ceil}{\lceil}{\rceil}
\DeclarePairedDelimiter{\floor}{\lfloor}{\rfloor}

\DeclareMathOperator*{\bernoulli}{\text{Bern}}
\DeclareMathOperator*{\binomial}{\text{Bin}}
\renewcommand{\L}{\mathcal{L}}
\DeclareMathOperator*{\loss}{\L}
% raised plus-minus
\newcommand{\rpm}{\sbox0{$1$}\sbox2{$\scriptstyle\pm$}
  \raise\dimexpr(\ht0-\ht2)/2\relax\box2 }
% :=
\newcommand*{\defeq}{\mathrel{\vcenter{\baselineskip0.5ex \lineskiplimit0pt
      \hbox{\scriptsize.}\hbox{\scriptsize.}}}%
  =}


\newcommand{\ensuretext}[1]{#1}
\newcommand{\marker}[2]{\ensuremath{^{\textsc{#1}}_{\textsc{#2}}}}

\newcommand{\draftcomment}[3]{\ensuretext{\textcolor{#3}{[#1 #2]}}}
% \renewcommand{\draftcomment}[3]{}  % uncomment for submission
\newcommand{\sam}[1]{\draftcomment{\marker{S}{T}}{#1}{orange}}


\newcommand{\term}[1]{\textbf{#1}} % term being defined
\newcommand{\seq}[1]{\mathbf{#1}}
\newcommand{\R}{\mathbb{R}}  % real numbers
\newcommand{\N}{\mathbb{N}}  % natural numbers
\newcommand{\Q}{\mathbb{Q}}  % rational numbers
\newcommand{\Z}{\mathbb{Z}}  % integers

\newcommand{\tabincell}[2]{\begin{tabular}{@{}#1@{}}#2\end{tabular}}

\newcommand{\rulesep}{\unskip\ \vrule\ }

\newcolumntype{L}[1]{>{\raggedright\let\newline\\\arraybackslash\hspace{0pt}}m{#1}}
\newcolumntype{C}[1]{>{\centering\let\newline\\\arraybackslash\hspace{0pt}}m{#1}}
\newcolumntype{R}[1]{>{\raggedleft\let\newline\\\arraybackslash\hspace{0pt}}m{#1}}

\newtheorem{theorem}{Theorem}
\newtheorem{lemma}[theorem]{Lemma}
\newtheorem{proposition}[theorem]{Proposition}
\newtheorem{corollary}[theorem]{Corollary}
\theoremstyle{definition}
\newtheorem{definition}[theorem]{Definition}
\newtheorem{example}[theorem]{Example}
\theoremstyle{remark}
\newtheorem{remark}[theorem]{Remark}

\newcommand*{\QEDA}{\hfill\ensuremath{\blacksquare}}
\newcommand*{\QEDB}{\hfill\ensuremath{\square}}

\newcommand{\pd}[2]{\frac{\partial #1}{\partial #2}}
